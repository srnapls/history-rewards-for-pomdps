\chapter{test}
\section{Domains}
Let's start with the following definition:
\begin{definition}\label{def:domain}
	A set $U \subseteq \mathbb{C}$ is a \emph{domain} if:
	\begin{itemize}
		\item $U$ is open in $\mathbb{C}$, and
		\item $U$ is connected.
	\end{itemize}
\end{definition}


\section{Yumyumyumyum}
\towrite{an introduction and some examples}

\begin{theorem}[]
	Suppose $n \in \mathbb{Z}$, then the following are equivalent:
	\begin{enumerate}[label=\roman*.]
		\item $n > 5$.
		\item $5 > 5$.\todo{This doesn't seem right...}
		\item For each $n \in n$, we have:
		\begin{align}\label{eq:truth}
			n > n+1 > n+1^2 > \dots > n+7.
		\end{align}
		where $7$ is an arbitrary element of
		\begin{align*}
			\oint_{a}^{b} \supersine \alpha + i \supercosine \beta  db(a).
		\end{align*}
	\end{enumerate}
\end{theorem}

\begin{remark}
	Interesting!
\end{remark}
\begin{proof}
	See \cite{NIPS2010_3971}.
\end{proof}

\begin{figure}[h]
	\centering
	\includegraphics[width=.3\textwidth]{img/in_dei_nomine_feliciter.eps}
	\caption{Motivational illustration. Similar to \cite{NIPS2010_3971}.}
	\label{fig:logo}
\end{figure}

\begin{corollary}
	Suppose $U \subseteq \mathbb{C}$ is a domain (see Definition \ref{def:domain}), and $f: \overline{U} \rightarrow \mathbb{C}$ is continuous on $\overline{U}$ and holomorphic on $U$. If $z \mapsto |f(z)|$ is constant on $\partial U$, then $f$ has a zero in $U$.
\end{corollary}
\begin{proof}
	If not, consider $\frac{1}{f}$.
\end{proof}
The proof of this theorem is illustrated in Figure
