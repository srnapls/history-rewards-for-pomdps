\towrite{introduction}
%add extra final state, in which we encode the actual transitions, the end state will be used to marked you are done. 

%if take product construction over only S and N without the s_f, o_f and and, we would get a reward every time we go accross that specific state, but! we only want reward when we are done.

\towrite{please not that all the text here is placeholder, just notes that need to be processed}

The resulting POMDP is a product construction between the original POMDP and the reward controller representing the history-based reward function. 

Since the reward in encoded in the states themself in $\mathcal{F}$, we dont want to just simple encode them in the POMDP in the state as well. If we were to do this, then we'd get the reward every time we passed over the state. We only want to obtain the relevant reward when we are \textit{finished} with the process. This is why we used the extra action \texttt{end} to mark the end of the process. Then, when we wish the process to end, we simply execute the action \texttt{end} and will then obtain the reward that was encoded in the relevant state from $\mathcal{F}$ where we finished upon. This new state $s_F$ only contains deterministic loops, ensuring that the process ends there. 

\section{Definition}
\begin{definition}
	\label{def:reward-mdp}
	The induced POMDP for reward controller $\mathcal{F}=(N, n_I, \Omega, \mathcal{R}, \delta, \lambda)$ on a POMDP $\mathcal{M}=(M,\Omega,O)$ where $M=(S,s_I,A,T_{M})$ is a tuple $\mathcal{M_\mathcal{F}}=(M_\mathcal{F},\Omega',O')$ where 
	\begin{itemize}
		\item $M_\mathcal{F} = (S',s'_I,A',T_{M_\mathcal{F}},\mathcal{R})$, the hidden MDP where:
		\begin{itemize}
			\item $S'=S\times N \cup \{s_F\}$, the finite set of states;
			\item $s'_I = \langle s_I, \delta(n_I, O(s_I))\rangle$, the initial state;
			\item $A'=A\cup \{\texttt{end}\}$, the finite set of actions;
			\item $T_{M_\mathcal{F}}:S'\times A' \to \Pi(S')$, the probabilistic transition function defined as:
				\begin{align*}				
					T_{M_\mathcal{F}}(s,\texttt{end},s_F) &= 1 \text{ for all } s\in S'\\
					T_{M_\mathcal{F}}(\langle s,n\rangle,a,\langle s',n'\rangle) &= \begin{cases}
						T_M(s,a,s') & \text{if } \delta(n,O(s')=n') \\
						0 & \text{otherwise}
					\end{cases}
				\end{align*}
			\item $\mathcal{R}:S'\times A'\times S' \to \mathbb{R}$ where 
                \begin{equation*}
					R(s,a,s') = \begin{cases}
					\sigma(n) & \text{if } a=\texttt{end} \text{ and } s=\langle s'',n\rangle \text{ and } s'=s_F\\
					0 & \text{otherwise} 
					\end{cases}
				\end{equation*}
		\end{itemize}
		\item $\Omega'=\Omega \cup \{o_F\}$, the observation spate
		\item $O':S'\to \Omega'$, the observation function where
			\begin{equation*}
				O'(s)= \begin{cases}
				O(s') & \text{if } s=\langle s',n\rangle\\
				o_F & \text{if } s=s_F
				\end{cases}
			\end{equation*}
		\end{itemize}
\end{definition}

Note that for the POMDP $\mathcal{M}$ we could only calculate the reward after we were done with the process. However, for the newly obtained POMDP $\mathcal{M}_{\mathcal{F}}$ we obtain the reward as the process continues, since it is now dependent only on the state and action.

\section{Implementation}
\towrite{the python part, where we simple combine the information of the reward controller together with the pomdp (without reward) together to create a new pomdp}

remove the end action from prism and code it into the pomdp in python. check if this is possible. 
\towrite{
the transformation to prism where we then add the extra \texttt{end} actions with the last state added.\\
- limit to $T$, which needs to be passed along \\
- for the end action, we only need to observation 
}

