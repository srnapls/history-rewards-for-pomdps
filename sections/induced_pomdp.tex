In this chapter we combine the obtained reward controller $\mathcal{F}$, which is based on the history-based reward function, together with the original POMDP. This allows us to remove the history-based aspect of the reward function, allowing us to calculate the reward step-by-step. This can be seen in \secref{s:induced_pomdp}. Firstly, since there is no method to ending the process we add another action to forcibly \textit{end} the process in \secref{s:extended_pomdp}. In \secref{s:induced_example} we will show an example. In \secref{s:induced_policy}. In \secref{s:induced_limit} we talk about on how to limit the computational process by only allowing sequences up to a certain length. In \secref{s:induced_implementation} we talk a bit about the implementation regarding this new induced POMDP.


\section{Extended POMDP}
\label{s:extended_pomdp}
Given a POMDP with has a history-based reward function, we want to obtain the related reward at any certain moment. But there is not a action in the model that ensures that the model stops and we can obtain the reward. 

This can be solved by adding an action to actively end the model and this can be done by extending the model with the action \texttt{end} together with a final state. This final state then should only consist of deterministic loops.

\begin{definition}
	The extended POMDP for a given POMDP $\mathcal{M}=(M,\Omega,O)$ where $M=(S,s_I,A,T_{M})$ is a new POMDP $\widetilde{\mathcal{M}} = (\widetilde{M},\Omega',O')$ where
		\begin{itemize}
		\item $\widetilde{M} = (S',s_I,A',T_{M_t})$, the hidden MDP where:
		\begin{itemize}
			\item $S'=S\cup \{s_F\}$, the finite set of states;
			\item $A'=A\cup \{\texttt{end}\}$, the finite set of actions;
			\item $T_{M_t}:S'\times A' \to \Pi(S')$, the probabilistic transition function defined as:
			\[ T_{M_t}(s,a,s') = \begin{cases}
						1 & \text{if } s'=s_F \text{ and } a = \texttt{end}\\
						T_M(s,a,s') & \text{otherwise} 
					\end{cases} \]
		\end{itemize}
	\item $\Omega'= \Omega\cup \{o_F\}$
	\item $O' : S' \to \Omega'$ where 
		\[ O'(s) = \begin{cases}
			o_F & \text{if } s=s_F \\
			O(s) & \text{otherwise} 
			\end{cases} \]
	\end{itemize}
	\label{d:extended_pomdp}
\end{definition}

We now also need to adjust the reward function for the extended POMDP, to accomodate for the new information.
\begin{definition}
	\label{def:reward-mdp}
	Given the extended POMDP $\widetilde{\mathcal{M}}$ and the original history-based reward function $R:\Omega^*\to \mathbb{R}$, we obtain the new reward funtion $\widetilde{R}:\Omega^*\times A\to\mathbb{R}$  where 
	\[\widetilde{R}(o_1 o_2\dots o_n,\texttt{end}) = 
		R(o_1 o_2 \dots o_{n})\]
\end{definition}


\section{Induced POMDP}
\begin{definition}
	\label{def:reward-mdp}
	The induced POMDP for reward controller $\mathcal{F}=(N, n_I, \Omega, \mathcal{R}, \delta, \lambda)$ on a POMDP $\mathcal{M}=(M,\Omega,O)$ where $M=(S,s_I,A,T_{M})$ is a tuple $\mathcal{M_\mathcal{F}}=(M_\mathcal{F},\Omega',O')$ where 
	\begin{itemize}
		\item $M_\mathcal{F} = (S',s'_I,A',T_{M_\mathcal{F}},\mathcal{R})$, the hidden MDP where:
		\begin{itemize}
			\item $S'=S\times N \cup \{s_F\}$, the finite set of states;
			\item $s'_I = \langle s_I, \delta(n_I, O(s_I))\rangle$, the initial state;
			\item $A'=A\cup \{\texttt{end}\}$, the finite set of actions;
			\item $T_{M_\mathcal{F}}:S'\times A' \to \Pi(S')$, the probabilistic transition function defined as:
				\begin{align*}				
					T_{M_\mathcal{F}}(s,\texttt{end},s_F) &= 1 \text{ for all } s\in S'\\
					T_{M_\mathcal{F}}(\langle s,n\rangle,a,\langle s',n'\rangle) &= \begin{cases}
						T_M(s,a,s') & \text{if } \delta(n,O(s')=n') \\
						0 & \text{otherwise}
					\end{cases}
				\end{align*}
			\item $\mathcal{R}:S'\times A'\times S' \to \mathbb{R}$ where 
                \begin{equation*}
					R(s,a,s') = \begin{cases}
					\sigma(n) & \text{if } a=\texttt{end} \text{ and } s=\langle s'',n\rangle \text{ and } s'=s_F\\
					0 & \text{otherwise} 
					\end{cases}
				\end{equation*}
		\end{itemize}
		\item $\Omega'=\Omega \cup \{o_F\}$, the observation spate
		\item $O':S'\to \Omega'$, the observation function where
			\begin{equation*}
				O'(s)= \begin{cases}
				O(s') & \text{if } s=\langle s',n\rangle\\
				o_F & \text{if } s=s_F
				\end{cases}
			\end{equation*}
		\end{itemize}
\end{definition}

Note that for the POMDP $\mathcal{M}$ we could only calculate the reward after we were done with the process. However, for the newly obtained POMDP $\mathcal{M}_{\mathcal{F}}$ we obtain the reward as the process continues, since it is now dependent only on the state and action.

\section{Example}
\begin{figure}[H]
	\centering
		\begin{tikzpicture}[node distance=1cm,on grid,auto]
			\node[initial,state] (0) 			{$s_0$};
			\node[dot]  (0a) [below=of 0] {};
			\node[dot]  (0b) [right=of 0] {};
			\node[state] (1) [right= 7cm of 0]		{$s_1$};
			\node[dot] (1a) [below=of 1] {};
			\node[dot] (1b) [above=of 1] {};
			\node[state,fill=black,text=white] (2) [below right= 5 cm of 0]		{$s_2$};
			\node[dot] (2c) [above=of 2] {};
			\path[-]    (0) edge [] node[right] {$a$} (0a)
                            edge [] node[above] {$b$} (0b)
                        (1) edge [] node[left]  {$a$} (1a)
                            edge [] node[left]  {$b$} (1b)
                        (2) edge [] node[right] {$c$} (2c);
			\path[->] (0a)  edge[bend left] 	node [left] {$0.5$}  (0)
			                edge[bend right] node [below] {$0.5$}  (2)
			          (0b)  edge[]              node [above] {$0.25$} (1)
			                edge[bend right]    node [right] {$0.75$} (2)
			          (1a)  edge[bend right]     node [right] {$0.25$} (1)
			                edge[bend left]    node [left]  {$0.75$} (2)
			          (1b)  edge[bend left]    node [right] {$0.25$} (1)
			                edge[bend right]     node [above] {$0.75$} (0)
			          (2c)  edge[bend right]     node [left] {$0.9$}  (2)
			                edge[]    node [right] {$0.1$}  (1);
	\end{tikzpicture}
\caption{POMDP with $\Omega=\{\square,\blacksquare\}$}
	\label{f:pomdp}
\end{figure}

\towrite{the reward controller involved}

\towrite{the finalized pomdp}

\section{Optimal policy}
\towrite{tikz graph thing w/ 4 blocks}

optimal policy $\pi$ for $\widetilde{\mathcal{M}_\mathcal{F}}$ also works for $\widetilde{\mathcal{M}}$. something with same observation sequences, for all actions.


to get the optimal policy for $\mathcal{M}$ we remove the end action. 

\section{Limiting the observation sequence}
\label{s:induced_limit}
However, when we try to calculate $P[[ \texttt{F}\:s=s_F]]$, this will not equate to 1 because there is no absolutely certainty that \texttt{end} will ever be performed. We can enforce the model to allow observation sequences up to a certain natural number $T$, such that every observation sequence $o_1 o_2 \dots o_n$ we know that $n\leq T$. 

\begin{definition}
We can extend the underlying MDP $M=(S\cup\{s_F\}, s_I, A, T_M)$ of the extended POMDP $\widetilde{M}=(M,\Omega,O)$ with some given counter $T$, creating a limited underlying MDP $(S',s'_I,A,T_M')$ where
\begin{itemize}
	\item $S' = S\times \{0,\dots, T\}$
	\item $s'_I = \langle s_I,0\rangle$
	\item $T_M':S'\times A \to \Pi(S')$ where 
	\[T_M'(\langle s_1,t_1\rangle,a,\langle s_2,t_2\rangle) = \begin{cases}
	T_M(s_1,a,s_2) & \text{ if } t_2 = t_1 + 1 \text{ and } t_2 \neq T\\
	1 & \text{ if } t_2 = T \text{ and } s_2 = s_F \\
	0 & \text{ otherwise }
	\end{cases}\]
\end{itemize}
\label{d:limited_pomdp}

The reward function $R:S'\times A\times S' \to \mathbb{R}$ will only look at the states of $S$ and will be transformed into
\[R(\langle s_1, t_1\rangle, a, \langle s_2, t_2\rangle) = R(s_1,a,s_2)\]
\end{definition}

In other words, we add a simple counter that keeps track of the number of actions allowed. At any point we are allowed to go to the final state, but when the counter reaches $T$, we force the model to enter the final state. This method enforces the process to always finish in the final state. \\

\towrite{this way the probability of reaching $s_F$ is one so we can calculate the r max to s ewual to $S_F$}

\section{Implementation}
\towrite{the python part, where we simple combine the information of the reward controller together with the pomdp (without reward) together to create a new pomdp}

remove the end action from prism and code it into the pomdp in python. check if this is possible. 
\towrite{
the transformation to prism where we then add the extra \texttt{end} actions with the last state added.\\
- limit to $T$, which needs to be passed along \\
- for the end action, we only need to observation 
}

