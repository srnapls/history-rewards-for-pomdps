\section{Definition}


\towrite{introduction}
The idea is to transform the history-based reward function into something more tangible. We transform it is so that we can obtain the reward per step instead of only at the end of a sequence.


Based on the history-based reward function $R:\Omega^*\to\mathbb{R}$ of a POMDP $\mathcal{M}$, we build a reward controller that mimics its behavior.

%todo: basically the same as a moore machine

%idea: when the word is done reading, you then obtain the reward. Normally when using a Moore machine you get the reward every time you enter it. However, since we won't be using the reward controller as a usual moore machine, it's fine to encodew it in the state. We will see in the next chapter how we then obtain the reward.

\begin{definition}
	A reward controller $\mathcal{F}$ is a reward machine $(N,n_I, \Omega, \mathbb{R}, \delta, \lambda)$, where
	\begin{itemize}
		\item $N$, the finite set of memory nodes;
		\item $n_I\in N$, the initial memory node;
		\item $\Omega$, the input alphabet;
		\item $\mathbb{R}$, the output alphabet;
		\item $\delta: N \times \Omega \to N$, the memory update;
		\item $\sigma: N \to \mathbb{R}$, the reward output. 
	\end{itemize}
\end{definition}

\towrite{transition to delta star}

\begin{definition}
We define $\delta^*:N\times\Omega^*\to N$ where $\delta^*(n,w)$ denotes the state we end up after reading word $\pi$ starting from state $n$ as follows
\begin{equation*}
\delta^*(n,\pi)=\begin{cases}
	n &\text{if } \pi=\epsilon \\
	\delta^*(\delta(n,o_1),o_2\dots o_n) & \text{if } \pi=o_1 o_2\dots o_n
	\end{cases}
\end{equation*}
\label{d:delta_star_rc}
\end{definition}


\section{From a list of sequences}
\label{sec:rc-sequences}
Let's say we are designing a model for an engineer and they want certain observation sequences to connect to a reward. Thus we are given a number of observation sequences $\texttt{seq}_1,\texttt{seq}_2,\dots,\texttt{seq}_n$ together with their associated real valued rewards $r_1,r_2,\dots,r_n$.
\begin{definition}
Given the observation sequences $\texttt{seq}_1,\texttt{seq}_2,\dots,\texttt{seq}_n$ and their associated rewards $r_1,r_2,\dots,r_n$ we define the history-based reward function $R:\Omega^*\to\mathbb{R}$, which we create as follows
\[R(w) = \begin{cases}
	r_i &\text{ if } w=\texttt{seq}_i \text { for } i\in \{1,\dots,n\} \\
	0   &\text{ otherwise}
	\end{cases}\]
	\label{d:created_reward_function}
\end{definition}
In $R$ we simply connect the observation sequence $\texttt{seq}_i$ to their respective reward $r_i$ and every other sequence is connected to zero.\\

We only want to obtain any of the rewards if their associated observation sequence has been observed in its entirety. Thus we create a reward controller in which we encode the reward in the node we end up in after reading the entire sequence. The idea is as follows: if we read the observation sequence and we end up in a certain node $n$, we obtain the reward $\sigma(n)$ in that node. It's important to note that if we, for example, have $R(\blacksquare\blacksquare)=2$ and $R(\blacksquare\blacksquare\square)=3$ and we read $\blacksquare\blacksquare\square$ we will only obtain reward $3$. 

Given all the sequences over which the Non-Markovian reward function is defined, let us create a reward controller through the following procedure. Note that we assume that all the sequences are unique.
\begin{algorithm}[H]
	\begin{algorithmic}[1]
		\Procedure{CreateRewardController}{sequences, $R$}
		\Require sequences
		\Require $R : \Omega^* \to \mathbb{R}$
		\State $n_I \gets $ \code{new Node()} \Comment{initial node} \label{l:initial_node}
		\State $n_F \gets $ \code{new Node()} \Comment{dump node}  \label{l:dump_node}
		\State \texttt{path}$(n_I)=\lambda$ \label{l:empty_path}
		\State $N\gets\{n_I,n_F\}$
		\ForAll{$\texttt{seq}=o_1 o_2\dots o_k$ in sequences}
			\State $n \gets n_I$
			\For{$i\gets 1,\dots, k$}
				\If{$\delta(n,o_i)$ is undefined}
					\State $n'\gets$ \code{new Node()} \Comment{create new memory node} \label{l:new_transition}
					\State \texttt{path}$(n)=o_1\dots o_i$
					\State $N \gets N \cup \{n'\}$		
					\State $\delta(n,o_i) \gets n'$	\label{l:set_transition}
				\EndIf 
				\State $n \gets \delta(n,o_i)$     \Comment{update memory node}
			\EndFor
			\State $\sigma(n) \gets R(\texttt{seq})$ \Comment{set reward} \label{l:set_reward}
		\EndFor
		\ForAll{$n\in N$} \Comment{makes $\delta$ and $\sigma$ deterministic}
			\ForAll{$o\in\Omega$}
				\If{$\delta(n,o)$ is undefined} \Comment{useless transition}
					\State $\delta(n,o) \gets n_F$ \label{l:self_loop}
				\EndIf
			\EndFor
			\If{$\sigma(n)$ is undefined}
				\State $\sigma(n) \gets 0$ \label{l:set_zero}
			\EndIf
		\EndFor
		\State \Return $(N,n_I,\Omega,\mathbb{R},\delta,\sigma)$
		\EndProcedure
	\end{algorithmic}
	\caption{Procedure for turning a list of sequences into a reward controller}
	\label{procedure:into_reward_controller}
\end{algorithm}

We start by creating an initial node in \lineref{l:initial_node} and a dump node in \lineref{l:dump_node}. The idea is that, since the reward controller is deterministic, if we need to determine the reward of a sequence that is (for example) longer than a known sequence (with reward), we don't want to end in the node in which the reward is encoded. Thus these zero-reward sequences are passed along to a node which will only consist of self-loops and will have a reward of zero encoded to them.

Then for every sequence which we are given, we walk through it. If we then come across a transition which isn't defined yet, we define it by making a new memory node in \lineref{l:new_transition}, adding it to $N$, and setting the transition to this new node. If the transition already existed, we simply update the memory node. After we are done with reading the sequence, we simply encode the reward into the node itself in \lineref{l:set_reward}.

Then since the reward controller needs to be deterministic, we set the other undefined values. Every other transition that hasn't been made yet, will be transferred to the dump node as mentioned above in \lineref{l:self_loop}. Furthermore, there are still nodes in which the reward is undefined. None of the given sequences ended up in these nodes, so per \defref{d:created_reward_function} we encode those to zero in \lineref{l:set_zero}.

Note that the set of nodes $N$ without $n_F$ together with the memory update function is represents a directed acyclic graph. This indicates for every node $n$ there is an unique path from the initial node $n_I$ to node $n$. This unique path is encoded in the function \texttt{path}$:N\setminus\{n_F\}\to\Omega^*$. This function is well-defined, since it's defined for $n_I$ in \lineref{l:empty_path}. Every other time a new node is neccesary, it is created in \lineref{l:new_transition}, and \texttt{path} is then immediately defined for the new node. This \texttt{path} function is needed for proving the following lemma. 


\begin{lemma}
For any sequence $\texttt{seq}\in\Omega^*$, let $r=R(\texttt{seq})$ be its associated reward. Then $\sigma(\delta^*(n_I,\texttt{seq}))=r$.
\begin{proof}
Given a sequence $\texttt{seq}$, we set $n$ to be the node we end up in, i.e. $n = \delta^*(n_I,\texttt{seq})$. \\
Now if $n=n_F$, we know that the associated reward is zero since $\sigma(n_F)=0$ per construction. A sequence can only end up in $n_F$ if it was not a part of the pre-defined sequences and following \defref{d:created_reward_function} the reward is then zero. \\
If $n\in N\setminus \{n_F\}$, we can obtain the unique path to node $n$ through $\texttt{path}(n)$. We know that this is equal to $\texttt{seq}$, so the associated reward is thus $R(\texttt{path}(n))=R(\texttt{seq})=r$.
\end{proof}
\label{lem:proof_seq}
\end{lemma}


We observe that the number of memory nodes $|N|$ of the newly created reward controller $\mathcal{F}$ is bounded by $|\Omega|^k + 1$, where $k=\max\limits_{seq\in sequences} |seq|$.\\
We acknowledge that this can lead to quite large reward controllers. Please note that using state of the art automata learning, the sizing of the reward controller obtained for a number of sequences can be decreased.

\subsection*{Example}
Say we are given the following sequences and rewards
\begin{enumerate}
	\item $\square\ \square\  $ with a reward of 15
	\item $\blacksquare\ \square\ \blacksquare\ $ with a reward of 20
	\item $\square\ \square\ \blacksquare\ \square\ $ with a reward of 12
	\item  $\blacksquare\ $ with a reward of 2
\end{enumerate}

Following the procedure~\ref{procedure:into_reward_controller} we create the associated reward controller. To show how the procedure works, we will show you the intermediate reward controller after processing every sequence. 

\subsubsection*{After sequence (1)}
\begin{figure}[H]
	\centering
		\begin{tikzpicture}[node distance=3cm,on grid,auto,shorten >=1pt,thick]
			\node[initial,state] (1) 				{$n_I$};
			\node[state] (2) [above right=of 1]		{$n_1$};
			\node[state] (3) [above right=of 2]		{$n_2/15$};
			\node[state] (9) [right=of 3]            {$n_F$};
			\path[->] 	(1) edge 	node [above] {$\square$}			(2)
						(2) edge 	node [above] {$\square$}			(3);
	\end{tikzpicture}
\caption{Reward controller after sequence (1)}
\end{figure}

\subsubsection*{After sequence (2)}
\begin{figure}[H]
	\centering
		\begin{tikzpicture}[node distance=3cm,on grid,auto,shorten >=1pt]
			\node[initial,state] (0) 				{$n_I$};
			\node[state] (1) [above right=of 0]		{$n_1$};
			\node[state] (2) [above right=of 1]		{$n_2/15$};
			\node[state] (3) [below right=of 0]      {$n_3$};
			\node[state] (4) [above right=of 3]      {$n_4$};
			\node[state] (5) [below right=of 4]      {$n_5/20$};
			\node[state] (8) [right=of 5]            {$n_F$};
			\path[->] 	(0) edge 	node [above] {$\square$}			(1)
							edge    node [above] {$\blacksquare$}	(3)
						(1) edge 	node [above] {$\square$}			(2)
						(3) edge    node [above] {$\square$}        	(4)
						(4) edge		node [above] {$\blacksquare$}	(5);
	\end{tikzpicture}
\caption{Reward controller after sequence (1) and (2)}
\end{figure}

\subsubsection*{After sequence (3)}
\begin{figure}[H]
	\centering
		\begin{tikzpicture}[node distance=3cm,on grid,auto,shorten >=1pt]
			\node[initial,state] (0) 				{$n_I$};
			\node[state] (1) [above right=of 0]		{$n_1$};
			\node[state] (2) [above right=of 1]		{$n_2/15$};
			\node[state] (3) [below right=of 0]      {$n_3$};
			\node[state] (4) [above right=of 3]      {$n_4$};
			\node[state] (5) [below right=of 4]      {$n_5/20$};
			\node[state] (6) [below right=of 2]      {$n_6$};
			\node[state] (7) [above right=of 6]      {$n_7/12$};
			\node[state] (8) [right=of 7]            {$n_F$};
			\path[->] 	(0) edge 	node [above] {$\square$}			(1)
							edge    node [above] {$\blacksquare$}	(3)
						(1) edge 	node [above] {$\square$}			(2)
						(2) edge		node [above] {$\blacksquare$}	(6)
						(3) edge    node [above] {$\square$}        	(4)
						(4) edge		node [above] {$\blacksquare$}	(5)
						(6) edge		node [above] {$\square$}			(7)
						;
	\end{tikzpicture}
\caption{Reward controller after sequence (1), (2) and (3)}
\end{figure}

\subsubsection*{After sequence (4)}
\begin{figure}[H]
	\centering
		\begin{tikzpicture}[node distance=3cm,on grid,auto,shorten >=1pt]
			\node[initial,state] (0) 				{$n_I$};
			\node[state] (1) [above right=of 0]		{$n_1$};
			\node[state] (2) [above right=of 1]		{$n_2/15$};
			\node[state] (3) [below right=of 0]      {$n_3/2$};
			\node[state] (4) [above right=of 3]      {$n_4$};
			\node[state] (5) [below right=of 4]      {$n_5/20$};
			\node[state] (6) [below right=of 2]      {$n_6$};
			\node[state] (7) [above right=of 6]      {$n_7/12$};
			\node[state] (8) [right=of 7]            {$n_F$};
			\path[->] 	(0) edge 	node [above] {$\square$}			(1)
							edge    node [above] {$\blacksquare$}	(3)
						(1) edge 	node [above] {$\square$}			(2)
						(2) edge		node [above] {$\blacksquare$}	(6)
						(3) edge    node [above] {$\square$}        	(4)
						(4) edge		node [above] {$\blacksquare$}	(5)
						(6) edge		node [above] {$\square$}			(7)
						;
	\end{tikzpicture}
\caption{Reward controller after sequence (1), (2) and (3)}
\end{figure}

\subsubsection*{Finalized Reward Controller}
Now we complete the reward controller by completing the rest of the transitions. Note that \texttt{path} was only used for proving \lemref{lem:reward_sequence}, so it is not included in any of the figures. 
In \figureref{f:final_reward_controller_regex} the dashed line denotes all the other possible letters for which the transition function $\lambda$ wasn't defined. 
\begin{figure}[H]
	\centering
		\begin{tikzpicture}[node distance=3cm,on grid,auto,shorten >=1pt]
			\node[initial,state] (0) 				{$n_I/0$};
			\node[state] (1) [above right=of 0]		{$n_1/0$};
			\node[state] (2) [above right=of 1]		{$n_2/15$};
			\node[state] (3) [below right=of 0]      {$n_3/2$};
			\node[state] (4) [above right=of 3]      {$n_4/0$};
			\node[state] (5) [below right=of 4]      {$n_5/20$};
			\node[state] (6) [below right=of 2]      {$n_6/0$};
			\node[state] (7) [above right=of 6]      {$n_7/12$};
			\node[state] (8) [right=of 7]            {$n_F/0$};
			\path[->] 	(0) edge 	node [above] {$\square$}			(1)
							edge    node [above] {$\blacksquare$}	(3)
						(1) edge 	node [above] {$\square$}			(2)
						(2) edge		node [above] {$\blacksquare$}	(6)
						(3) edge    node [above] {$\square$}        	(4)
						(4) edge		node [above] {$\blacksquare$}	(5)
						(6) edge		node [above] {$\square$}			(7)
						;
			\path[->,dashed] (1) edge [bend right]		node {} (8)
							 (2) edge [bend left]		node {} (8)
							 (3) edge [bend right]		node {} (8)
							 (4) edge [bend right]		node {} (8)
							 (5) edge [bend right]		node {} (8)
							 (6) edge		node {} (8)
							 (7) edge 		node {} (8)
							 (8) edge[loop above] node  {}   ();	\end{tikzpicture}
\caption{Final reward controller}
\label{f:final_reward_controller_regex}
\end{figure}

\subsection*{Implementation}
In \href{https://gitlab.science.ru.nl/srietbergen/thesis/-/blob/master/code/reward_controller_seq.py}{\texttt{code/reward\textunderscore controller\textunderscore seq.py}} we have the following:\\

\texttt{reward\textunderscore controller\textunderscore from\textunderscore sequences(sequences,omega)}\\
Given a dictionary in the shape of $\{\texttt{seq}_1 : r_1,\:\texttt{seq}_2 : r_2,\dots,\: \texttt{seq}_n : r_n\}$ together with the set $\Omega$ representing the input alphabet, it will return a reward controller as described in \algorithmref{procedure:into_reward_controller}.

\section{From regular expressions}
\label{sec:rc-regex}

\towrite{introduction into why using regular expressions for rewards}

Given a number of regular expressions over observations defined as $e_1,e_2,\dots,e_n$ together with their respective rewards $r_1,r_2,\dots,r_n\in\mathbb{R}$. Let us define a reward function $R$ that maps the regular expression to their respective reward, in other words $R(e_i)=r_i$. \\

We want to create a reward controller that mimics the behaviour of several regular expressions and their associated rewards. Note that we only want a reward when the sequence of observations is accepted by the language generated by the regular expression. The first step would be is to create a DFA that is generated by the regular expression given. This can be done through simply turning the regular expression into a Non-Deterministic Finite Automaton (with $\lambda$-transitions) and then turning that into a DFA or using other known methods\cite{p:regex-to-dfa}. All that is left for a single regular expression is to keep track of the rewards associated to their final states.\\

So given the $n$ regular expression, we create $n$ DFAs. Let $D_i=(Q_i,q_{0,i},\Omega, \delta_i,F_i)$ be the DFA that accepts the language generated by $e_i$. And then per construction we have that $L(D_i)=L(e_i)$. \\

Note that since we want to obtain a reward controller, we have to encode the reward in the nodes. This is solved by only encoding the reward of DFA $D_i$ in all states of $F_i$. For example if $\texttt{seq}\in\Omega^*$ gets accepted by $D_i$, we have to make sure that the state it ends up in - i.e. the final state(s) - has the reward encoded in its state(s). This is done by the following definition.
\begin{definition}
Let $R_A:Q_1\cup Q_2\cup\dots\cup Q_n\to\mathbb{R}$ be a function that maps any state $q$ of all the state spaces of $D_1,D_2,\dots,D_n$ to their respective rewards. If $q$ is a final state of DFA $D_i$ it should get the reward corresponding to the regular expression used for that specific DFA. In other words,
\begin{equation*}
R_A(q) = \begin{cases}
R(e_i) & \text{ if } q\in F_i \\
0 & \text{ otherwise }
\end{cases}
\end{equation*}
\label{d:associated_reward}
\end{definition}

Having obtained all these seperate DFAs, we can now create a DFA that will accept any word that is accepted by any of the seperate DFAs as follows.

\begin{definition}
The induced product DFA for given DFAs $D_1,D_2,\dots,D_n$ where $D_i=(Q_i,q_{0,i},\Sigma,\delta_i,F_i)$ is a tuple $D=(Q,q_0,\Sigma,\delta,F)$ where 
\begin{itemize}
\item $Q = Q_1 \times Q_2 \times \dots \times Q_n$
\item $q_0 = \langle q_{0,1}, q_{0,2}, \dots, q_{0,n}\rangle$
\item $\Sigma$, the same input alphabet
\item $\delta(\langle q_1,q_2,\dots,q_n\rangle,a)= \langle \delta_1(q_1,a), \delta_2(q_2,a),\dots,\delta_n(q_n,a)\rangle$
\item $F=\{\langle q_1,q_2,\dots,q_n\rangle \mid \exists i \in \{1,2,\dots,n\} : q_i\in F_i\}$
\end{itemize}
\label{d:product_automaton}
\end{definition}

\begin{lemma}
Given $n$ DFAs where $D_i=(Q_i,q_{0,i}, \Sigma,\delta_i,F_i)$, let $D$ be the product automaton as obtained in \defref{d:product_automaton}. Then we $L(D)=L(D_1)\cup L(D_2)\cup \dots L(D_n)$.
\begin{proof}
	\begin{flalign*}
		w\in L(D) &\Longleftrightarrow \delta_N^*(q_0,w)\in F \\
		& \Longleftrightarrow \langle \delta_1^*(q_{0,1},w), \delta_2^*(q_{0,2},w),\dots,\delta_n^*(q_{0,n},w)\rangle\in F \\
		&\Longleftrightarrow \exists i\in\{1,\dots,n\} :  \delta_i^*(q_{0,i},w)\in F_i \\
		&\Longleftrightarrow \delta_1^*(q_{0,1},w)\in F_1 \text{ or } \delta_2^*(q_{0,2},w)\in F_2 \text{ or } \dots \text{ or } \delta_n^*(q_{0,n},w)\in F_n \\
		& \Longleftrightarrow w \in L(D_1) \text{ or } w\in L(D_2) \text{ or } \dots \text{ or } w\in L(D_n)\\
		&\Longleftrightarrow w\in L(D_1)\cup L(D_2)\cup \dots \cup L(D_n)
	\end{flalign*}
\end{proof}
\end{lemma}

The only step left to obtain the reward controller is to connect the obtained product DFA together with the associated rewards of the states.

\begin{definition}
Given a (product) DFA $N=(Q,q_0,\Omega,\delta,F)$ and the associated reward function $R_A$, we define the induced reward controller $\mathcal{F}=(N,n_I,\Omega, \mathbb{R}, \delta_\mathcal{F},\sigma)$ as follows
\begin{itemize}
\item $N=Q$
\item $n_I=q_0$
\item $\delta_\mathcal{F}=\delta$
\item $\sigma: Q\to\mathbb{R}$ where $\sigma(\langle q_1,q_2,\dots, q_n\rangle) = \sum\limits _{i=1}^n R_A(q_i)$
\end{itemize}
\label{d:reward_controller_regex}
\end{definition}

Note that the $\sigma$ is defined by taking the sum over the associated rewards. This is because if we have a sequence $\texttt{seq}\in\Omega^*$ that is accepted by several regular expressions given, it should then obtain all the seperate rewards associated with those regular expressions. Through the following lemma we ensure that for any sequence $\texttt{seq}\in\Omega^*$ the reward controller obtains the combination of rewards depending on the final state after having read $\texttt{seq}$.

\begin{lemma}
Given $e_1,e_2,\dots,e_n$ a sequence of regular expression together with their associated rewards $r_1,r_2,\dots,r_n$, let $D$ be the product automaton as defined in \defref{d:product_automaton} build from the DFAs $D_i$ for which $L(D_i)=L(e_i)$. Then let $\mathcal{F}=(N,n_I,\Omega,\mathbb{R},\delta,\sigma)$ be the reward controller as defined in \defref{d:reward_controller_regex} given $D$. We say that for all possible words $\texttt{seq}\in\Omega^*$ the following holds:
\[\sigma(\delta^*(n_I,\texttt{seq}))=\sum\limits_{e\in\{e_i\mid\texttt{seq}\in L(e)\}}R(e_i)\]
\begin{proof}
\begin{flalign}
\sigma(\delta^*(n_I,\texttt{seq})) &= \sigma(\langle q1,q2,\dots,q_n\rangle)  \label{p:r_l1} \\
	&= \sum\limits_i^n R_A(q_i)\label{p:r_l2}\\
	&= \sum\limits_{\substack{i\in\{1,\dots,n\}\\ q_i\in F_i}} R_A(q_i) \label{p:r_l3}\\
&= \sum\limits_{\substack{i\in\{1,\dots,n\}\\ q_i\in F_i}} R(e_i) \label{p:r_l4}\\
&= \sum\limits_{\substack{i\in\{1,\dots,n\}\\ \delta^*(q_{0,i},\texttt{seq}) \in F_i }} R(e_i)\label{p:r_l5}\\
&= \sum\limits_{\substack{i\in\{1,\dots,n\}\\ \texttt{seq}\in L(D_i)}} R(e_i)\label{p:r_l6}\\
&= \sum\limits_{\substack{i\in\{1,\dots,n\}\\ \texttt{seq}\in L(e_i)}}R(e_i)\label{p:r_l7}\\
&= \sum\limits_{e\in\{e_i\mid\texttt{seq}\in L(e_i)\}} R(e)\label{p:r_l8}
\end{flalign}
For \equref{p:r_l1} we simply use \defref{d:delta_star_rc} and the fact that $D$ is deterministic, so it ends up in an unique state after reading $\texttt{seq}$.
For \equref{p:r_l2} we use the definition for $\sigma$ as seen in \defref{d:reward_controller_regex}. For \equref{p:r_l3} we use that fact that in \defref{d:associated_reward} we observe that $R_A(q_i)$ is equal to zero if $q_i\notin F_i$ and only produces a non-zero value for all $q_i\in F_i$. Thus we only look at the $q_i$ which return a non-zero value. Since we now know we only look at the non-zero reward values, we can use \defref{d:associated_reward} again in \equref{p:r_l4}. From \defref{d:delta_star} we can rewrite the equation in \equref{p:r_l5}. For \equref{p:r_l6} we use \defref{d:accepted_language}. Since per construction $L(e_i)=L(D_i)$ for all $i\in\{1,\dots,n\}$, we rewrite the term in \equref{p:r_l7}. Finally in \equref{p:r_l8} we simply rewrite the term under the sum.
\end{proof}
\label{lem:proof_regex}
\end{lemma}

Logically, the size of the reward controller obtained through a series of $n$ automata is bounded by $|D_1||D_2| \dots  |D_n|$, where $D_i$ is the automata obtained through the regular expressions $e_i$. When implementing this, make sure to minimize the automata whenever possible to decrease the sizing of the reward controller. \todo{implementation relevant?}

\subsection*{Example}
Let's say we are given 2 regular expressions. One is that an even number off $\square$ gives a reward of $10$ and the other states that an odd number of $\blacksquare$ gives a reward of $15$. In other words 
$R(e_1)=R(\texttt{even number of }\square)=10$ and $R(e_2)=R(\texttt{odd number of }\blacksquare)=15$\\

Let us first obtain the two DFAs that are generated by $e_1$ and $e_2$. Those can be seen in \figureref{f:e1_e2}. 
\begin{figure}[H]

\begin{subfigure}[H]{0.4\textwidth}
	\centering
		\begin{tikzpicture}[node distance=3cm,on grid,auto]
			\node[initial,state,accepting] (1) 				{$q_{0,1}$};
			\node[state] (2) [right=of 1]		{$q_{1,1}$};
			\path[->] 	(1) edge[bend left] 	node [above] {$\square$}			(2)
							edge[loop above] node [above] {$\blacksquare$}   ()
						(2) edge[bend left] 	node [above] {$\square$}			(1)
							edge[loop above] node [above] {$\blacksquare$}   ();
	\end{tikzpicture}
\caption{DFA for regular expression even number of $\square$}
\end{subfigure}
\hfill
\begin{subfigure}[H]{0.4\textwidth}
	\centering
		\begin{tikzpicture}[node distance=3cm,on grid,auto]
			\node[initial,state] (1) 				{$q_{0,2}$};
			\node[state,accepting] (2) [right=of 1]		{$q_{1,2}$};
			\path[->] 	(1) edge[bend left] 	node [above] {$\blacksquare$}			(2)
							edge[loop above] node [above] {$\square$}   ()
						(2) edge[bend left] 	node [above] {$\blacksquare$}			(1)
							edge[loop above] node [above] {$\square$}   ();
	\end{tikzpicture}
\caption{DFA for regular expression for odd number of $\blacksquare$}
\end{subfigure}
\caption{}
\label{f:e1_e2}
\end{figure}
Then we create the product automaton as defined in \defref{d:product_automaton}. The result can be seen in \figureref{f:pd}. 

\begin{figure}[H]
	\centering
		\begin{tikzpicture}[node distance=3cm,on grid,auto]
			\node[initial,state,accepting] (1) 				{$\langle q_{0,1}, q_{0,2} \rangle$};
			\node[state] (2) [right=of 1]		{$\langle q_{1,1}, q_{0,2} \rangle$};
			\node[state,accepting] (3) [below=of 1]		{$\langle q_{0,1}, q_{1,2} \rangle$};
			\node[state,accepting] (4) [right=of 3]		{$\langle q_{1,1}, q_{1,2} \rangle$};
			\path[->] 	(1) edge[bend left] 	node [above] {$\square$}			(2)
							edge[bend left] 	node [left] {$\blacksquare$}			(3)
						(2) edge[bend left] 	node [above] {$\square$}			(1)
							edge[bend left] 	node [left] {$\blacksquare$}			(4)
						(3) edge[bend left] 	node [left] {$\blacksquare$}			(1)
							edge[bend left] 	node [above] {$\square$}			(4)
						(4) edge[bend left] 	node [left] {$\blacksquare$}			(2)
							edge[bend left] 	node [above] {$\square$}			(3);
	\end{tikzpicture}
\caption{Product DFA for both regular expressions}
	\label{f:pd}
\end{figure}

From this we then obtain the reward controller as per \defref{d:reward_controller_regex}, and can be found in \figureref{f:rc}. Note that 
\begin{flalign*}
R_A(q_{0,1})&=10\\
R_A(q_{1,1})&= R_A(q_{0,2})=0\\
R_A(q_{1,2})&=15
\end{flalign*}

\begin{figure}[H]
	\centering
		\begin{tikzpicture}[node distance=4cm,on grid,auto]
			\node[initial,state] (1) 			{$\langle q_{0,1}, q_{0,2} \rangle /10$};
			\node[state] (2) [right=of 1]		{$\langle q_{1,1}, q_{0,2} \rangle /0$};
			\node[state] (3) [below=of 1]		{$\langle q_{0,1}, q_{1,2} \rangle /25$};
			\node[state] (4) [right=of 3]		{$\langle q_{1,1}, q_{1,2} \rangle /15$};
			\path[->] 	(1) edge[bend left] 	node [above] {$\square$}			(2)
							edge[bend left] 	node [left] {$\blacksquare$}			(3)
						(2) edge[bend left] 	node [above] {$\square$}			(1)
							edge[bend left] 	node [left] {$\blacksquare$}			(4)
						(3) edge[bend left] 	node [left] {$\blacksquare$}			(1)
							edge[bend left] 	node [above] {$\square$}			(4)
						(4) edge[bend left] 	node [left] {$\blacksquare$}			(2)
							edge[bend left] 	node [above] {$\square$}			(3);
	\end{tikzpicture}
\caption{Reward Controller for $R$}
	\label{f:rc}
\end{figure}

\subsection*{Implementation}
We used \cite{g:regex-to-dfa} as a base for creating a DFA from a given regular expression. The regular expression needs to have the following grammar (this can also be seen in \texttt{code/regex-to-dfa/grammar/RegEx.g4}):
\begin{alltt}
prog : (regex newline)*;

regex : regex '*'      #kleene-star
  | regex regex	       #concatenation
  | regex '|' regex    #alternation
  | ID         	       #identifier
  | '\(\lambda\)'	   #epsilon
  | '(' regex ')'      #parenthesis
  ;

newline : '\(\backslash\)n{}';

ID: [a-zA-Z0-9];
WS: [\(\backslash\)t\(\backslash\)r ]->skip;
\end{alltt}

In \href{https://gitlab.science.ru.nl/srietbergen/thesis/-/blob/master/code/reward_controller_regex.py}{\texttt{code/reward\textunderscore controller\textunderscore regex.py}} we have the following:\\

\texttt{rename(D)}\\
Transform the given reward controller $D$ into one that ensures that the states are labeled with numbers from $0$ to $|D|-1$. \\

\texttt{regex\textunderscore to\textunderscore dfa(regex, omega)}:\\
Given a regular expression conform to the syntax as defined above and the input language $\Omega$ over which it is defined, we transform it into a DFA.\\

\texttt{union(machines, rewards)}:\\
Given a list $n$ of \texttt{machines} ($D_1,D_2,\dots, D_n$) and a list or $n$ rewards ($r_1,r_2,\dots, r_n$), we create the induced product DFA according to \defref{d:product_automaton}. Then we (create and) return the induced reward controller as defined in \defref{d:reward_controller_regex}.\\

\texttt{reward\textunderscore controller\textunderscore from\textunderscore regex(info,omega):}:\\
Given $n$ regular expressions together with their associated reward in dictionary ($\{e_1 :r_1, e_2:r_2, \dots e_n:r_n\}$), together with the input language $\Omega$ specified, we return the reward controller representing the information given. 

