\section*{Set Theory}
Let $S$ be any countable set, then $|S|$ denotes the cardinality. We let $S^*$ and $S^\omega$ denote the set of finite and infinite sequences over $S$, respectively. For a sequence $\pi\in S^*$ we can denote the length by $|\pi|$.

Let an alphabet $\Sigma$ be a a finite set consisting of letters. A word is defined as a sequence of letters $w=w_1 w_2\dots w_n\in \Sigma^*$. A language $L$ is a subset of all possible words given an alphabet $\Sigma$, so $L\subseteq \Sigma^*$. Let $\epsilon$ denote the empty word, so $|\epsilon|=0$.

A regular language is a language that can be defined by a regular expression. The language accepted by a regular expressions $e$ is denoted as $L(e)$.



\section*{Probability Theory}
For any countable set $S$ we can define a \textit{discrete probability distribution} as $\psi: S\to[0,1]$ where $\sum_{s\in S} \psi(s)=1$. The set of all possible probability distributions over $S$ is denoted as $\Pi(S)$. We denote the support of a \textit{probability distribution} as $supp(\psi)=\{s\in S\mid \psi(s)>0\}$.\\

\towrite{random variable, expected value}
%- expected value of a discrete random variable E[X]

	


